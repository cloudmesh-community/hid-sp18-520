% status: 0
% chapter: TBD

\title{Amazon Relational Database Services (RDS)}


\author{Arijit Sinha}
\affiliation{%
	\institution{Indiana University}
	\country{USA}}
\email{arisinha@iu.edu}


% The default list of authors is too long for headers}
\renewcommand{\shortauthors}{A.Sinha}

\keywords{hid520, hid-sp18-520, i524, Amazon RDS, RDS, AWS RDS}

\maketitle

\section{Overview}

Amazon Relational Database Service also known as RDS is a prime ``web service 
provided on cloud by Amazon to operate with relational databases. 
With AWS database migration services, it provides an mechanism for 
replicating/ migrating any existing databases on AWS cloud''
~\cite{hid-sp18-520-amazonrds}.
Amazon Relational database services has been effectively used for  handling
and managing relational databases, which in return provides high performance, 
security, maximum availability and compatibility. 
It is compatible with variety of database engines running in background 
including ``Amazon Aurora, PostgrSQL, MySQL, MariaDB, Oracle and 
Microsoft SQL Server''~\cite{hid-sp18-520-amazonrds}.

\section{Key features}

Amazon RDS provides many key features which makes it popular are such as below - 
1. Scaling Storage - It can automatically increase the storage once size or 
volume of the databases ins reaching its maximum capacity.
2. Offers less Administrative workload - When the services are getting setup, 
all the databases instances are configured with its respective database engines. 
Amazon provides command line and management consoles for easy administration 
of the databases.
3. Reliability - It can replicate the data to a standby instance on different 
Availability Zone using the Multi-AZ DB instance. It provides  automates backups, 
user defined snapshots of the data stored on Amazon S3. In the event of failure 
of an hardware, it can automatically replace the instance.
4. High Performance and Secure -  It provides high performance using General 
purpose SSD storage and Provisioned IOPS SSD storage. It provides the feature of 
encrypting the databases using keys (AWS Key Management Services). Along with, it 
provides Amazon VPC for network isolation for databases on cloud to securely 
connect with on premise applications
~\cite{hid-sp18-520-amazonrds}.
 

\section{Building Blocks for Amazon RDS}

DB Instances are Amazon RDS primary building blocks which is a secured database 
environment on AWS cloud. As mentioned in above section, these DB instance can be 
easily managed using simple API, AWS management console and Command line interfaces 
to set the configurations and monitor the behavior and capabilities of relational 
databases. It does not need any additional database maintenance software.
In these DB instances, we can have multiple databases created by many users or 
applications.
In the background, we have DB engines interacting with DB instances. Few of the 
examples can be MySQL, Maria DB, PostgreSQL, Oracle and Microsoft SQL Server DB 
engines.
There are 3 types of storage available with DB instances (Magnetic, General 
Purpose SSD and Provisioned IOPS).
Storage capacity depends on various storage type and respective database engines 
it been configured.
Amazon RDS can select IP address range, subnets, access control list and 
configure routing to make it more secure and reliable.
Another component provided by Amazon is IAM (Identity and Access Management), 
which this you can provide provision on users to create, delete, modify read 
any DB instances
~\cite{hid-sp18-520-amazonrdswel}.

\section{RDS Automated and Manual Monitoring Tools}

Amazon RDS can monitored for it performance and can be reported in case of 
any issues or failures on DB instances, DB clusters DB Cluster Snapshots, DB 
parameter group or DB security group.
On real time, DB instances or clusters can be monitored. It also maintains the 
database log files which can referred or consulted in cases of any failure or 
issues encountered.
Amazon TRDS also provided extra feature for monitoring with CloudWatch for 
metrics, alarms and logs, along with service health status.
With Command Prompt - using below command can view performance metrics and 
alarm- 

``aws cloudwatch list-metrics --namespace AWS/RDS

put-metric-alarm''

With API - using the CloudWatch API GetMetricStatistics with start and end time 
can provide detail metrics on performance and form setting up alarm 
with PutMetricAlarm on DB Instance''.

Based on the user defined baseline for performance and resource to be monitored 
Amazon RDS store the respective monitoring logs including your CPU, RAM, Disk 
Space consumption.
Number of users connected to database can be monitored with kind of operation 
getting performed on Database with Amazon RDS console.  

Various Amazon RDS metrics dimensions can be on name of the Engine, specific 
DB Instances, DB Clusters with Roles and database class
~\cite{hid-sp18-520-amardsmon}.

\bibliographystyle{ACM-Reference-Format}
\bibliography{report} 